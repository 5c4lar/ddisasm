
\section{Experiments}
\label{sec:experiments}
  \begin{itemize}
    \item
   Three compilers (versions): gcc 5.4, gcc 8, and clang 3.8
\item
  Five optimization flags: None, -O1, -O2, -O3 and -Os
  \item
    Three benchmarks: Coreutils, (a subset of) CGC, Real world examples
  \end{itemize}
  The list of real world examples is in Table\ref{table:real_examples}.
  \begin{table}
    \caption{Real world examples}
    \label{table:real_examples}
    \begin{tabular}{llllll}
      grep-2.5.4 &
      gzip-1.2.4  &
      bar-1.11.0  &
      conflict-6.0 &
      ed-0.2 &
      ed-0.9 \\ 
      marst-2.4 &
      units-1.85 &
      doschk-1.1 & 
      bool-0.2 &
      m4-1.4.4 &
      patch-2.6.1 \\ 
      enscript-1.6.1 &
      bison-2.1 &
      sed-4.2 & 
      flex-2.5.4 &
      make-3.80 & 
      rsync-3.0.7 \\
      gperf-3.0.3 & 
      re2c-0.13.5 & 
      lighttpd-1.4.18 &
      lighttpd-1.4.11 &
      tar-1.29 &
    \end{tabular}
  \end{table}

  \begin{table}
\caption{Funtionality tests}
    \begin{tabular}{llllll}
    Benchmark       & Examples    & Combinations   & Succeed   & Failed  & \%\\ \hline
    Coreutils       & 106         & 1590           & 1590      &   0     & 100\% \\
    CGC             & 86          & 1290           & 1268      & 18      & 98.2\% \\
    Real world & 23 & 345 & 342 & 3 & 99\% \\
  \end{tabular}
\end{table}
  The binaries are stripped (with the flag \texttt{-strip-unneeded})
  before rewriting. Once the assembler code has been obtained, the
  result has is stirred as follows: for each instruction, between 1
  and 10 nop instructions are added randomly with 1/3 probability
  throughout the code.


  In many of the examples that fail, the original binary produces
  different test results in different executions. So it is quite possible that
  the binaries are not broken.

  \subsection{Performance}
  Fig.\ref{fig:perf} contains the performance results of the analysis
  for the different benchmarks and the relation between analysis time
  and the size of the binaries.
  
  \begin{figure}
    \begin{tabular}{@{}l@{}l}
   \multicolumn{2}{l}{Performance of Datalog for coreutils}\\
      \begin{tikzpicture}
        \begin{axis}[
            width=0.50\textwidth,
            xlabel=Size (bytes),
            ymax=20,
            ylabel=Datalog time (seconds)]
          \addplot+[
            only marks,
            scatter,
            mark size=2pt]
          table[x index=3,y index=2]
          {stats_coreutils.csv};
        \end{axis}
      \end{tikzpicture}
      &
      \begin{tikzpicture}
        \begin{axis}[
             width=0.50\textwidth,
             xlabel=Size (bytes),
             ymax=20,
            ylabel=Total time (seconds)]
          \addplot+[
            only marks,
            scatter,
            mark size=2pt]
          table[x index=3,y index=0]
          {stats_coreutils.csv};
        \end{axis}
      \end{tikzpicture}\\
      Performance in CGC examples\\
      
        \begin{tikzpicture}
          \begin{axis}[
              width=0.50\textwidth,
              xlabel=Size (bytes),
              ymax=40,
              ylabel=Datalog time (seconds)]
            \addplot+[
              only marks,
              scatter,
              %  mark=halfcircle*,
              mark size=2pt]
            table[x index=3,y index=2]
            {stats_CGC.csv};
          \end{axis}
        \end{tikzpicture}
        &
        \begin{tikzpicture}
          \begin{axis}[
              width=0.50\textwidth,
              xlabel=Size (bytes),
              ymax=40,
              ylabel=Total time (seconds)]
            \addplot+[
              only marks,
              scatter,
              %  mark=halfcircle*,
              mark size=2pt]
            table[x index=3,y index=0]
            {stats_CGC.csv};
          \end{axis}
        \end{tikzpicture}\\
Performance in real world examples\\
     \begin{tikzpicture}
       \begin{axis}[
           width=0.50\textwidth,
           xlabel=Size (bytes),
           ymax=55,
          ylabel=Datalog time (seconds)]
        \addplot+[
          only marks,
          scatter,
          mark size=2pt]
        table[x index=3,y index=2]
        {stats_real.csv};
      \end{axis}
    \end{tikzpicture}
    &
    \begin{tikzpicture}
      \begin{axis}[
          width=0.50\textwidth,
          xlabel=Size (bytes),
          ymax=55,
          ylabel=Total time (seconds)]
        \addplot+[
          only marks,
          scatter,   
          mark size=2pt]
        table[x index=3,y index=0]
        {stats_real.csv};
      \end{axis}
    \end{tikzpicture}
  \end{tabular}
    \caption{Performance graphs. From top to bottom: Coreutils, CGC,
      and real world examples.  The datalog processing time on the
      left-hand side and the total processing time on the right-hand
      side.}
    \label{fig:perf}
\end{figure}
